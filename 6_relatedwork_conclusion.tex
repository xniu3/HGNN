\section{Related Work}\label{section 2}
% There have been many works that utilize GCNs to extract local features to preserve high-order structure patterns. 
% \cite{AbuElHaija2019MixHopHG} mixes first-order and high-order feature representations of neighbors repeatedly in each layer. 
% \cite{motif2019} selects a few motifs for each node for aggregation. 
% \cite{Jin_Song_Shi_2020} captures complex structure patterns by anonymous walks. 
% \cite{topic2020} uses topic models to pre-select anchors to discover representative structure patterns and guide the aggregation. 

Hyperbolic geometric space was introduced into complex networks earlier to represent the small world and scale-free of complex networks~\cite{Krioukov2010Hyperbolic,papadopoulos2012popularity}. 
In the real world, complex networks are generally tree-like networks and hyperbolic geometric space is used to represent two important properties of such networks: the exponential expansion of network scale and the network hierarchy~\cite{Krioukov2010Hyperbolic}. 
With high capacity and hierarchical structure-preserving ability of hyperbolic space, it is recently introduced into graph neural networks~\cite{HGNN_Qi,HGCN_ChamiYRL19,HAT}. 
To the best of our knowledge, only a few studies have considered the balance of topological structure and feature information of graph data. 

\section{Conclusion}
In this work, we proposed RAHGNN, a novel adaptive node embedding framework by reinforcement learning in hyperbolic geometric space, and we practice this framework based on the hyperboloid model. 
For graphs of different hierarchical topologies, the adaptively selected curvature can provide a good trade-off and fusion between the hierarchy and features of different graph data. 
Moreover, the curvature selection can also intuitively give a reasonable interpretation of model learning, reflecting the model preference to capture information between the structure and features of a graph. 
The proposed RAHGNN is evaluated on datasets with different hierarchical topologies for different tasks, and the experiment results show the effectiveness of our approach. 