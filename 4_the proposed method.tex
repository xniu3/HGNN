\section{Reinforcement Learning Based Adaptive Hyperbolic Graph Neural Network}\label{section 4}

In this section, we introduce the proposed RAHGNN, an Adaptive Hyperbolic Graph Neural Network, for node hyperbolic representation learning. 
The overall architecture of RAHGNN is shown in Figure~\ref{framework}. 
We design the HGNN module to learn the node representation in hyperbolic space with a given curvature, and design the adaptive curvature exploration module to explore the curvature of hyperbolic space which can get the maximum reward reward from downstream task.
Then, we present a collaborative reinforcement learning framework with two agents  Hyperbolic Graph Neural Network Agent (HGNN-Agent) and the Adaptive Curvature Exploration agent (ACE-Agent) satisfying Nash equilibrium. 

\subsection{Hyperbolic Graph Neural Network Agent}
The goal of Hyperbolic Graph Neural Network Agent (HGNN-Agent) is to learn the fusing representation of graph structure and node features with a given curvature. 
It focuses on how to integrate the structure information and node features of a graph. 

\subsubsection{HGNN with Variable Curvature}
Though many hyperbolic embedding methods using hyperboloid (Lorentz) models have proved their success in feature representation, they are mostly based on the assumption that the hyperboloid has a constant curvature. 
Therefore, to achieve our goal, we need to extend the hyperbolic graph neural network into a variable curvature version.
A hyperbolic graph neural network layer with variable curvature consists of the three operations as follows:

\paragraph{Hyperbolic Activation with Variable Curvature}
For graph node Euclidean embeddings $x = (x_1, x_2, \dots, x_n) \in \mathcal{R}^{n} $, we first transform the $n$-dimensional vector to the $(n+1)$-dimensional space in pseudo-polar coordinate as follows:
\begin{equation}\label{equ:7}
    x \rightarrow x_{\mathcal{L}}^{K} = (\zeta \left\|x\right\|,\frac{x}{\zeta \left\|x\right\|}).
\end{equation}
Then, the exponential map $\mathrm{exp}^{K}_{x}:\mathcal{E}^n \rightarrow \mathcal{L}^{n+1}$ with curvature $K$ can be defined as: 
\begin{equation}\label{equ:11}
    \mathrm{exp}^{K}_{\mathcal{L}}: \mathbf{h}^{\zeta}_{x \mathcal{L}}=(\cosh{\zeta \left\|\mathbf{h}_x\right\|}, \sinh{\zeta \left\|\mathbf{h}_x\right\|} \frac{\mathbf{h}_x}{\zeta \left\|\mathbf{h}_x\right\|}), 
\end{equation}
where $\mathbf{h}^{\zeta}_{j \mathcal{L}}$ is the projection on hyperboloid of a node hidden state $h_v$. 

For the hyperboloid (Lorentz) model, we consider the Lorentzian scalar product of a Riemannian manifold with curvature $K$ is:
\begin{equation}\label{equ:6}
    \langle x,x \rangle_\mathcal{L} : K x_0^2 + \sum_{i=1}^{n}{x_i^2} = K, 
\end{equation}
where $x_0 = \sqrt{1-\frac{x^2}{K}}$. 
To make the calculation easier, we linearly approximate $x$ in Equation~\eqref{equ:6} to $x = \|x\|/\zeta$, where the $\zeta$ is the curvature parameter, given by $K = -\zeta^2 < 0, \zeta > 0$.

Since we have given the variable curvature version of hyperbolic distance between node $i$ and node $j$ with curvature $K$, and the distance between the node-pair $(i, j)$ on hyperboloid by Equation~\eqref{equ:2}:
\begin{equation}\label{equ:Hdist}
    \mathbf{d}^{K}_{\mathcal{L}}(i,j) = \mathrm{arccosh}{(- \langle \mathbf{h}^{\zeta}_{i \mathcal{L}},\mathbf{h}^{\zeta}_{j \mathcal{L}})\rangle}_{\mathcal{L}}).
\end{equation}

\paragraph{Hyperbolic Attention Aggregation with Variable Curvature}
We next discuss how to conduct the adaptive hyperbolic attention mechanism to aggregate node features by appropriate curvature. 

The GNNs models usually use convolution layers or attention layers to aggregate node feature information. 
As discussed in the previous Section~\ref{section 3}, our goal is to obtain an aggregation with variable curvature in hyperbolic space. 
Therefore, we extend the hyperbolic attention network~\cite{HAtt}, and present a hyperbolic attention aggregation with variable curvature through the following steps.

\textbf{Step-1: }
Considering the variable curvature, given the feature vectors $\mathbf{h}_i$ and $\mathbf{h}_j$ of nodes $i$ and $j$ in Euclidean space, we perform self-attention on the nodes, and the hyperbolic attention weight $\alpha\left(\mathbf{h}_{i}, \mathbf{h}_{j}, K \right)$ can be calculated as follows:
\begin{equation}
    \alpha\left(\mathbf{h}_{i}, \mathbf{h}_{j}, K \right)=\mathrm{exp}\left(-\beta \mathbf{d}^{K}_{\mathcal{L}}(i,j)-b\right),
\end{equation}
where $\pi(\mathbf{h}_{i})$ is the hyperbolic embedding given by Equation~\eqref{equ:7}, and $\mathbf{d}^{\zeta}_{\mathrm{H}}(\cdot,\cdot)$ is the variable curvature hyperbolic distance given by Equation~\eqref{equ:Hdist}. 
Then, we transform these node embeddings $\mathrm{exp}^{K}_{\mathcal{L}}(\mathbf{h})$ to the Klein model, which can represent the node embedding in the interior of the unit disk by Equation~\eqref{equ:11}. 
The hyperbolic embedding of node $v$  in Klein model is defined as:
\begin{equation}
    \mathbf{h}^{\zeta}_{x \mathcal{L}} \rightarrow \mathbf{h}_{x \mathcal{K}}^{\zeta} = \frac{\sinh(\zeta \left\|\mathbf{h}_x\right\|) \mathbf{h}_x}{\zeta \left\|\mathbf{h}_x\right\| \cosh(\zeta \left\|\mathbf{h}_x\right\|)}. 
\end{equation}

\textbf{Step-2:} Then we can utilize the Einstein midpoint in hyperbolic space to compute the weight sum operation in Euclidean space: 
\begin{equation}
    \mathbf{h}_{i \mathcal{K}}^{\zeta} = \sum_{j}\left[\frac{\alpha_{i j} \gamma\left(\mathbf{h}_{j \mathcal{K}}^{\zeta}\right)}{\sum_{\ell} \alpha_{i \ell} \gamma\left(\mathbf{h}_{j \mathcal{K}}^{\zeta}\right)}\right] \mathbf{h}_{j \mathcal{K}}^{\zeta},
\end{equation}
where the Lorentz factors is $\gamma(\mathbf{h}_{j \mathcal{K}}^{\zeta})=\frac{1}{\sqrt{1-  \frac{\sinh^2(\zeta  \left\|\mathbf{h}_j\right\|)}{\cosh^2(\zeta \left\|\mathbf{h}_j\right\|)}    } }$. 

\paragraph{Riemannian Optimization with Variable Curvature}
Since the curvature is a variable parameter, the optimizer needs to be extended accordingly. 
In the hyperboloid model, Riemannian-SGD (RSGD)~\cite{DBLP:conf/icml/NickelK18} is widely used for optimization. 
In the following, we also present a variable curvature version of RSGD with curvature parameter $\zeta$. 
\begin{equation}
\begin{aligned}
    \mathbf{h}^t & = g_{\theta_t}^{-1} \nabla f(\theta), \\ 
    \mathrm{grad}f(\theta_t) & = f(\mathbf{h}^t) + \frac{\langle \mathbf{h}^t,\theta_t \rangle_{\mathcal{L}}}{\zeta^2}\theta_t,  \\ 
    \theta_{t+1} & = \mathrm{exp}_{\theta_t}(-\eta \mathrm{grad}f(\theta_t)).
\end{aligned}
\end{equation}
The initialized embeddings $\mathbf{h}^t$ are close to the origin of $\mathcal{H}^n$ by sampling from the uniform distribution $\mathcal{U}(-0.001, 0.001)$ and by setting $x_0 = \sqrt{1+\frac{x^2}{\zeta^2}}$ according to Equation~\eqref{equ:6}.

\subsubsection{HGNN Agent}
Given a graph $G$ with the adjacency matrix $A$ and the Euclidean feature $X$, the first layer of HGNN maps from Euclidean to hyperbolic space, then the HGNN stacks multiple hyperbolic graph attention layers.  
Each hyperbolic graph attention layer transforms and aggregates neighbor’s embeddings in hyperboloid model with different curvature, and outputs the hyperbolic embeddings as results.
In the HGNN-Agent, the action is defined as whether to update the learned embeddings by taking the new curvature, and the reward is defined directly as performance changes for downstream tasks.

\subsection{Adaptive Curvature Exploration Agent}
Since the curvature determined the distance metric in hyperbolic space, it is difficult to update it as a learning parameter using back-propagation in the HGNN model. 
Moreover, because of the properties of hyperbolic logarithmic mapping and Riemannian optimization, the updating of trainable curvature parameters is limited to a very small range during the learning process. 
This problem makes it difficult to learn a good curvature parameter (In the Section~\ref{section 5}, we verify the performance of HGNN at different curvatures).
To address this issue, we design an Adaptive Curvature Learning Agent (ACE-Agent) which utilizes the reinforcement learning algorithm to explore the optimal curvature parameter for each hyperbolic attention layer adaptively. 
We design the learning module independently for the curvature parameter in each layer through the following two steps. 

\textbf{Step-1: }Considering the computational efficiency and a wider exploration space, we define the action $a_{C}$ as adding or subtracting a fixed value $\Delta\zeta \in [0,1]$ to the curvature parameter. 
We adopt an $\epsilon-greedy$ policy with an exploration probability $\epsilon$. 

\textbf{Step-2: }Then we project the embeddings $\mathbf{h}^{t-1}_{H}$ of the $t-1$ epoch as the input into the hyperboloid manifold with the new curvature parameters:
\begin{equation}
    \boldsymbol{\mathbf{h}^t_{C}} \gets \mathrm{exp}^{K}_{\mathcal{L}}(\mathbf{h}^{t-1}_{H}, \zeta^{t}_{C}), 
\end{equation}
where $\mathrm{exp}^{K}_{\mathcal{L}}$ is the exponential mapping, $\boldsymbol{\mathbf{h}^{t-1}_{H}}$ is the embeddings of last epoch, and $\zeta^{t}_{C}$ is the new curvature parameter from \textbf{Step-1}. 
We input $\mathbf{h}^t_{C}$ to the downstream task and get the reward. 

\subsection{RAHGNN Architecture with Nash Q-Learning}
In this section, we summarize the collaborative learning of the HGNN agent and the Curvature Learning agent based on reinforcement learning.

Our goal is to converge the learning of the two agents to Nash equilibrium. 
In other words, a single agent cannot update the learning result himself to improve the accuracy of the collaborative results in the downstream task at a Nash equilibrium.
Similarly, we define the state, reward, and termination for the entire framework as follows: 
\paragraph{State.} 
The state $s^t$ at epoch $t$ is the current state of the hyperbolic embedded space. 
Here we use the last updated optimal curvatures of two agents as states $s^t$:
\begin{equation}
    \mathbf{s}^t = (\zeta^t_{H}, \zeta^t_{C}),
\end{equation}
where $\zeta^t_{H}$, $\zeta^t_{C}$ are the current optimal curvatures of HGNN-Agent and ACE-Agent, respectively.  

\paragraph{Action.} 
We define the collaborative action $\mathbf{a}^t = (a^t_{H}, a^t_{C})$ for both agents based on current state. 
The HGNN Agent decides whether to update the learned embeddings by taking the curvature from the ACE-Agent. 
The ACE-Agent decides whether to adjust the curvature parameter.

\paragraph{Reward.} 
Since the HGNN agent and ACE-Agent are independent in the learning process, we define the reward function $reward(s_e, a_e)$ for each $a^t$ at $s^t$ directly based on the task performance:
\begin{equation}
\begin{aligned}
    r^t = reward\left(s^t, a^t\right)= (r^t_{H}, r^t_{C}), 
\end{aligned}
\end{equation}
where $r^t_{H}$ and $r^t_{C}$ are rewards of the HGNN agent and ACE-Agent at epoch $t$, respectively. 

\paragraph{Termination.} 
If the HGNN-Agent and the ACE-Agent have reached Nash equilibrium, the RL algorithm will stop, and the curvature parameter $\zeta$ will keep fixed during the next training process. 
\begin{equation}
\begin{aligned}
    \pi_{H}^*(s'), \pi_{H}^*(s')_{C} \gets Nash(Q_{H}(s'), Q_{C}(s')).
\end{aligned}
\end{equation}
It means that the RL algorithm finds the optimal curvature. 

Since the optimization problem is to find the Nash equilibrium point of the two agents, we use Nash Q-learning~\cite{hu2003nash} to learn the MDP. 
The Nash Q-learning optimization fits the Bellman optimality equation as below: 
\begin{equation}
\begin{aligned}
    NashQ_i(s') ~~=~~ & \pi_{H}^*(s') \pi_{C}^*(s') Q_i(s'),\\
    Q_i(s, a^t_{H}, a^t_{C}) \gets & Q_i(s, a^{t}_{H}, a^t_{C}) \\
    & + \alpha(r^i_t + \gamma NashQ_i(s') - Q_i(s, a^{t}_{H}, a^t_{C})). 
\end{aligned}
\end{equation}

In RAHGNN, the accuracy as the reward is given by the different downstream tasks. 
For link prediction, we use the Fermi-Dirac decoder~\cite{Krioukov2010Hyperbolic,NickelK17Poincare} to compute the connection probability between two nodes:
\begin{equation}
\begin{aligned}
    prob(i,j) = \left [ e^{ (\mathbf{d}^{K}_{\mathcal{L}}(i,j)^2 - r) / t} + 1 \right ]^{-1}, 
\end{aligned}
\end{equation}
where $\mathbf{d}^{K}_{\mathcal{L}}$ is the hyperbolic distance in Equation~\eqref{equ:Hdist} and $r$, $t$ are hyper-parameters.
For node classification, we directly classify nodes on the hyperboloid manifold using the hyperbolic multi-class logistic regression~\cite{HNN:GaneaBH18}.

% \begin{algorithm}
% \LinesNumbered
% \renewcommand{\algorithmicrequire}{\textbf{Input:}}
% \renewcommand{\algorithmicensure}{\textbf{Output:}}
% \caption{RAHGNN: Adaptive Hyperbolic Graph Neural Network by Multi-Agent Reinforcement Learning} 
%     \begin{algorithmic}[1] 
%         \Require Feature vectors and labels; Number of Epochs $E$; Initial curvatures for HGNN encoding layers $\boldsymbol{\zeta_0}$; 
%         \Ensure Embeddings for the feature vectors.  
    
%         \State Initialize RAHGNN model with $\boldsymbol{\zeta_0}$ and set RL environment;
        
%         // Train model
    
%         \For{$t = 1, 2, \cdots, E$}
%             \State Choose curvatures $(\zeta^t_{H}, \zeta^t_{C})$ based on current state; 
%             \State Set $s' = (\boldsymbol{\zeta^{t}_{H}, \zeta^{t}_{C}})$ using $\epsilon-Greedy$ policy;
    
%             // Steps in RL environment with model training\\
%             // Embedding, accuracy and reward of HGNN-Agent 
%             \State $\boldsymbol{\mathbf{h}^t_{H}} \gets HGNN \ encoding \ output$
%             \State $acc^{t}_{H} \gets Decode(\boldsymbol{\mathbf{h}^{t}_{H}})$
%             \State $r_H^t \gets \Delta acc^{t}_{H} = acc^{t}_{H} - acc^{t-1}_{H}$
            
%             // Embedding, accuracy and reward of ACE-Agent
%             \State $\boldsymbol{\mathbf{h}^t_{C}} \gets \mathrm{exp}^{K}_{\mathcal{L}}(\boldsymbol{\mathbf{h}^{t-1}_{H}}, \zeta^{t}_{C})$
%             \State $acc^{t}_{C} \gets Decode(\boldsymbol{\mathbf{h}^t_{C}})$
%             \State $r_C^t \gets \Delta acc^{t}_{C} = acc^{t}_{C} - acc^{t-1}_{C}$
            
%             // Calculate best policy for each agent
%             \State $\pi_{H}^*(s'), \pi_{C}^*(s') \gets Nash(Q_{H}(s'), Q_{C}(s'))$
%             \For{$i \in \{H, C\}$}
            
%                 // Update both agents using Nash Q-learning
%                 \State $NashQ_i(s') = \pi_{H}^*(s') \pi_{C}^*(s') Q_i(s')$ 
%                 \State $Q_i(s, a^t_{H}, a^t_{C}) \gets Q_i(s, a^{t}_{H}, a^t_{C}) + \alpha(r_i^t + \gamma NashQ_i(s') - Q_i(s, a^{t}_{H}, a^t_{C}))$
%             \EndFor
%             \State Save $\boldsymbol{\mathbf{h}^t_{H}}$ when $acc^t_H$ hits the best;
%             \State Rollback to last epoch's model if $acc^{t}_{H}$ continue to decline;
%         \EndFor
%     \end{algorithmic}
% \end{algorithm}


\begin{algorithm}[t]
\LinesNumbered
\caption{RAHGNN: Adaptive Hyperbolic Graph Neural Network by Multi-Agent Reinforcement Learning} 
\KwIn{Feature vectors and labels; Number of Epochs $E$; Initial curvatures for HGNN encoding layers $\boldsymbol{\zeta_0}$.}
\KwOut{Embeddings for the feature vectors.}
Initialize RAHGNN model with $\boldsymbol{\zeta_0}$ and set RL environment;\\
\tcp{Train model;}
\For{$t = 1, 2, \cdots, E$}{
    Choose curvatures $(\zeta^t_{H}, \zeta^t_{C})$ based on current state;\\
    Set $s' = (\boldsymbol{\zeta^{t}_{H}, \zeta^{t}_{C}})$ using $\epsilon-Greedy$ policy;\\
    % \tcp{Steps in RL environment with model training;}
    \tcp{Embedding, accuracy and reward of HGNN-Agent;}
    $\boldsymbol{\mathbf{h}^t_{H}} \gets HGNN \ encoding \ output$\\
    $acc^{t}_{H} \gets Decode(\boldsymbol{\mathbf{h}^{t}_{H}})$\\
    $r_H^t \gets \Delta acc^{t}_{H} = acc^{t}_{H} - acc^{t-1}_{H}$\\
    \tcp{Embedding, accuracy and reward of ACE-Agent;}
    $\boldsymbol{\mathbf{h}^t_{C}} \gets \mathrm{exp}^{K}_{\mathcal{L}}(\boldsymbol{\mathbf{h}^{t-1}_{H}}, \zeta^{t}_{C})$\\
    $acc^{t}_{C} \gets Decode(\boldsymbol{\mathbf{h}^t_{C}})$\\
    $r_C^t \gets \Delta acc^{t}_{C} = acc^{t}_{C} - acc^{t-1}_{C}$\\
    \tcp{Calculate best policy for each agent;}
    $\pi_{H}^*(s'), \pi_{C}^*(s') \gets Nash(Q_{H}(s'), Q_{C}(s'))$\\
    \For{$i \in \{H, C\}$}{
        \tcp{Update both agents using Nash Q-learning}
        $NashQ_i(s') = \pi_{H}^*(s') \pi_{C}^*(s') Q_i(s')$\\
        $Q_i(s, a^t_{H}, a^t_{C}) \gets Q_i(s, a^{t}_{H}, a^t_{C}) + \alpha(r_i^t + \gamma NashQ_i(s') - Q_i(s, a^{t}_{H}, a^t_{C}))$\\
    }
    Save $\boldsymbol{\mathbf{h}^t_{H}}$ when $acc^t_H$ hits the best;\\
    Rollback to last epoch's model if $acc^{t}_{H}$ continue to decline;
}
\end{algorithm}


